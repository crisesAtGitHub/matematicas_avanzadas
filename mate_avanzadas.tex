\documentclass[landscape]{article}
\usepackage{forest}
	\usepackage[left=1cm,right=1cm,top=1cm,bottom=1cm,a0paper]{geometry}
    \usepackage{tikz}
    \usetikzlibrary{graphs,quotes}
	\usetikzlibrary{arrows}

\tikzset{
  treenode/.style = {align=center, inner sep=0pt, text centered,
    font=\sffamily},
  arn_n/.style = {treenode, circle, white, font=\sffamily\bfseries, draw=black,
    fill=black, text width=3em},% arbre rouge noir, noeud noir
  arn_r/.style = {treenode, circle, red, draw=red, 
    text width=1.5em, very thick},% arbre rouge noir, noeud rouge
  arn_x/.style = {treenode, rectangle, draw=black,
    minimum width=0.5em, minimum height=0.5em}% arbre rouge noir, nil
}

\begin{document}
    
    \begin{forest}
for tree={%
    l sep=1cm,
    s sep=0.2cm,
    minimum height=0.8cm,
    minimum width=1cm,
    draw %Put lines around each
    },
        box/.style={rounded corners, draw, fill=gray!20},
        line/.style={draw, thick, <-}
        test/.style n args={3}{align=center, content={#1 \\#2 \\#3}}
    [LOS ANIMALES, box [Se clasifican por[FORMA, box[Pueden ser[Redondos,circle,draw[Como[Los erizos de mar, box]]][Alargados,circle,draw[Como[Lagartos, box]]][Sin patas,circle,draw[Como[Caracol\textrm{,} lombriz, box]]][Con muchas patas,circle,draw[Como[Cien pies,box]]]]][ALIMENTACI\'ON,box[Pueden ser[Herb\'ivoros, circle, draw[Se alimentan de[Plantas,box]]][Carn\'ivoros, circle, draw[Se alimentan de[Otros animales,box]]][Omn\'ivoros,circle, draw[Se alimentan de[Plantas y otros animales,box]]][Mam\'iferos,circle,draw[Se alimentan de[Leche,box]]]]][TAMA\~NO,box[Pueden ser[Grandes,circle,draw[Como[Elefant\\Cebra Rinoceronte, box]]][Peque\~nos,circle,draw[Como[,box,]]]]]]]
    \end{forest}
    \newpage
    \begin{forest}
    for tree={%
    l sep=1cm,
    s sep=0.2cm,
    minimum height=0.8cm,
    minimum width=1cm,
    draw %Put lines around each
    },
        box/.style={rounded corners, draw, fill=gray!20},
        line/.style={draw, thick, <-}
        test/.style n args={3}{align=center, content={#1 \\#2 \\#3}}
        [Matem\'aticas Avanzadas para la Ingenier\'ia, box
			[Variable compleja,circle]
			[Series de Fourier,circle]
			[Transformadas Integrales,circle]        
        ]
    \end{forest}
    \newpage
\begin{forest}[
	[funci\'on impulso unitario]
	[funciones peri\'odicas]
	[transformada r\'apida de fourier]
	[transformada discreta de fourier]
	[transformada z]
	[transformada z discreta]
	[ecuaciones en diferencias]
	[convoluci\'on circular]
	[delta dirac]
	[escal\'on unitario (funci\'on de Heaviside)]
	[exponenciales complejas]
	[convoluci\'on en $\ell^2$
		[conmutativa]
		[distributiva]
		[Asociativa]
		[funciones en $\ell^1$]	
	]
	[funciones en $L^2$]
	[transformada de Fourier
		[propiedades]	
	]
	[Sucesiones de funciones]
	[funciones ortogonales]
	[coeficientes de fourier]
	[desigualdad de Cauchy-Schwartz]
	[Forma compleja de la serie de Fourier]
	[f\'ormula de De' Moivre]
	[serie de fourier trigonom\'etrica]
	[teorema de parseval]
	[pulso rectangular]
	[pulso triangular]
	[convergencia de la serie de Fourier]
	[teorema de parseval]
	[c\'alculo de transformadas]
	[convoluci\'on en el tiempo discreto]
	[Transformada de Fourier funci\'on impulso]
	[convergencia de serie de Fourier]
]
\end{forest}

\begin{enumerate}
\item 	funci\'on impulso unitario
\item	funciones peri\'odicas
\item	transformada r\'apida de fourier
\item	transformada discreta de fourier
\item	transformada z
\item	transformada z discreta
\item	ecuaciones en diferencias
\item	convoluci\'on circular
\item	delta dirac
\item   escal\'on unitario (funci\'on de Heaviside)
\item	exponenciales complejas
\item	convoluci\'on en $\ell^2$
\item		conmutativa
\item		distributiva
\item		Asociativa
\item		funciones en $\ell^1$
\item	funciones en $L^2$
\item	transformada de Fourier
\item		propiedades
	\begin{enumerate}
		\item Desplazamiento en el tiempo
		\item Desplazamiento de frecuencia
		\item Derivaci\'on en integraci\'on
		\item 
	\end{enumerate}		
\item	Sucesiones de funciones
\item	funciones ortogonales
\item	coeficientes de fourier
\item	desigualdad de Cauchy-Schwartz
\item	Forma compleja de la serie de Fourier
\item	f\'ormula de De' Moivre
\item	serie de fourier trigonom\'etrica
\item	teorema de parseval
\item	pulso rectangular
\item	pulso triangular
\item	convergencia de la serie de Fourier
\item	teorema de parseval
\item	c\'alculo de transformadas
\item	convoluci\'on en el tiempo discreto
\item 	Convoluci\'on en el tiempo
\item 	Convoluci\'on en frecuencia
\item   Interpretaci\'on gr\'afica de la convoluci\'on
\item	Transformada de Fourier funci\'on impulso
\item	convergencia de serie de Fourier
\item   transformada r\'apida de fourier
\end{enumerate}


\end{document}